% Chapter 2
\chapter{Literature Review} % Write in your own chapter title
\label{LitReview}
%\lhead{} % Write in your own chapter title to set the page header

\section{Introduction}
WSN technology has matured such that it is a core component of the developing Internet of Things (IoT). The IoT describes the way in which networking capabilities formerly restricted to more traditional computers are being extended to physically-embedded devices, allowing for the acquisition and transfer of data on a much larger scale. WSN are geographically clustered networked devices equipped with one or more transducers which can range from a simple temperature probe to an implanted medical device. In general, these devices have limited computational resources and operate on supplies with limited power and energy. Due to these limitations, developers need to consider the design constraints of WSN applications. Many WSN devices are constrained such that their computational resources may not support a complete operating system (OS), which have been  fundamental to traditional computing systems. The lack of a traditional OS may complicate the development or limit the functionality of an application.

The range of devices being added to the IoT which generate data grows daily. Even though extending the operational lifespan and overcoming computational limitations remains at the forefront of the challenges facing WSN technology, numerous developments are enabling  greater WSN functionality. Some of the features being developed within the domain of WSN are intelligent routing and communication protocols, fault tolerance and network health monitoring, service oriented design, and application execution frameworks. In recent years, developers and researchers have streamlined many of these features and have created new abstractions for WSNs. These features allow developers to overcome many of the hurdles facing modern WSNs. 


\section{WSN Execution}
The functionality necessary for WSN devices includes two critical features: distributed sensing and wireless data aggregation. WSN nodes have the ability to sample data from transducers and the ability to enact changes through the use of actuators. Ideally, researchers would be able to automate their experiments through an application program that interacts with individual or groups of WSN nodes. Clever use of abstraction to manage the complexities of the WSN will dictate how an application is executed; an application might run as an executable binary directly on a sensor node's hardware or perhaps within an abstraction of the network at a higher layer in the form of scripted events or command sequences. WSN hardware is often extremely restrictive in its computational resources, and therefore application deployment and execution must be carefully designed.\\ 

Ivester et al.\ in~\cite{Ivester} describe ISEE, an interactive execution framework for monitoring network services and specifying operation of applications executing on a WSN node using a design paradigm known as service oriented architecture (SOA). Network monitoring functionality is an important set of features, especially as experiments increase in complexity. The generalization of these features into modular services allows them to be used with greater versatility and independent of specific application code. Researchers who have developed other modern WSN platforms have also adopted the SOA paradigm of implementing WSN functionality. Hammoudeh et al.\ in~\cite{Hammoudeh} describe a SOA approach to WSN fault tolerance and inter-node cooperation.\\

Another category of WSN execution paradigms includes those that use a virtual machine (VM). A virtual machine emulates a particular hardware architecture by translating instructions to another hardware architecture, thus allowing software designed for a particular platform to execute atop a different platform. Two of the more prominent WSN reconfiguration approaches that utilize virtual machines are Mat\'e ~\cite{Levis} and MagnetOS ~\cite{Barr}. Mat\'e is a virtual machine that is created to overcome many of the challenges of implementing heterogeneous network execution on resource-constrained wireless nodes. The fundamental feature of Mat\'e is a byte-code interpreter. Programs and  applications, even those with minimal complexity, can be hundreds of kilobytes in size. The virtual machine takes high-level operational instructions and organizes them into capsules that can be smaller than 100 bytes in size. The Mat\'e byte code interpreter running on individual nodes allows for the execution of capsules that are disseminated throughout the network. This allows for a scalable approach for specifying network execution on a distributed system of resource constrained nodes. Similarly, the creators of MagnetOS use a virtual machine to specify network-level execution. These researchers use a version of the Java Virtual Machine (JVM) designed to appear as though it is operating on a single computer; however, the computer is actually a WSN. This abstraction of the WSN as a platform which can run the JVM allows for Java application code to be written by developers. The Java objects produced by the JVM can be sent throughout the network to different physical hosts, enabling the distributed execution of a single Java application. This is performed by a byte-code level translation into instructions that the nodes can execute. Mat\'e and MagnetOS can compress large programs into small device operations, due to the abstractions which they were able to make in their design. Though a complete virtual machine abstraction of a WSN is not utilized in the work of this thesis, the core principles of energy and resource aware network execution through abstraction is one of the primary design goals. \\

Flikkema et al.~\cite{Flikkema} describe the design and implementation of a cyber-physical system which utilizes an ultra low power sensing platform, streaming data middleware, and a control-oriented approach to performing complex ecological experiments. Several core principles from the previous works such as energy-aware network execution and abstracted application development through high-level programming languages have been designed into this platform. An important distinction between previously described works and the platform used in this thesis is that the WSN cyber-infrastructure is a network of networks. An individual cluster of nodes at a geographic location form a WSN, but behavior needs to be tracked and controlled across multiple WSN locations. This platform forms the foundation of the work of this thesis and will be described in greater detail in Chapter 3.

As the complexity of WSNs and diversity of applications increase, so increases researchers' need to reconfigure specific hardware components or WSN properties. A simple sensing network might only support unidirectional communication where the nodes pass their data to a base station. More sophisticated sensing networks support bi-directional communication where nodes both send and receive information. Bidirectional communication is a critical feature for any WSN system to support remote device reconfiguration and reprogramming. Depending on the size of a WSN supporting a particular application, a change could affect one node or even an entire cluster of nodes, depending on the type or magnitude of the change. For example, a researcher monitoring temperature fluctuations over a large geographic area with a cluster of WSN nodes might want to increase the sampling rate to observe fine-scale environmental responses to a weather event. The difficulty of this task depends on how device reconfiguration was designed into the WSN platform and accompanying cyber-infrastructure. \\

Reprogramming and reconfiguration are often used interchangeably, as they might have the same effect on a researcher's experiment, but there is an important distinction between them. According to some interpretations, like e.g. ~\cite{Eronu}, if changes are made exclusively to the software components, then the device has been reprogrammed. Alternatively, if there is some change in the device hardware, the device is said to have been reconfigured. For the work of this thesis, we will define  reprogramming as involving transfer of new source code or application software to one or more devices, whereas reconfiguration implies that software parameters or hardware functionality can be altered without the addition of new program code. Ivester et al.~\cite{Ivester} describe the role of control information in a WSN as a means for reconfiguration.  Control information describes commands or network state information that are injected into a WSN, causing change in one or more nodes. Software called Ismanage used in ~\cite{Ivester} accompanies ISEE by providing several features, the most important of which is the ability to disseminate control information into the WSN for the purpose of reprogramming and reconfiguration. This approach is similar to the reconfiguration methodology used in WiSARDNet, the platform used for this thesis.\\ 

Eronu et al.~\cite{Eronu} summarize many of the issues and challenges surrounding WSN reconfiguration. Given that WSN hardware is often limited in power availability, the energy overhead involved in WSN reconfiguration is a major concern, especially when the number of nodes to be reconfigured increases~\cite{Eronu}. The authors of ~\cite{Reinhardt} describe a wireless data collection protocol named ORiNoCo, designed specifically to utilize the existing bidirectional messaging capabilities of a WSN for the purpose of energy efficient network reconfiguration. Efficient network reconfiguration is also a core feature of the Mat\'e and MagnetOS platforms discussed in the previous section ~\cite{Levis},~\cite{Barr}. Nodes that run the Mat\'e byte-code converter execute capsules which contain small control sequences. In this way, new configurations can be encoded and transmitted to a WSN node. The target node, upon receiving the encoded configuration, can decode the capsule with an intermediate software layer that translates the capsule's sequence into specific values to be stored in the device's memory. The reconfiguration of nodes in the MagnetOS platform allows for software changes to be instigated within an image that runs on the Java Virtual Machine. The virtual machine then interprets the image and sends the appropriate byte-code values to each hardware device in the network. Reconfiguration of these nodes then becomes no more difficult than executing a new application. We can see that when the sensing hardware is made available as a service to higher layers as in~\cite{Ivester}, and bidirectional communication protocols are leveraged effectively, network-wide reconfiguration of many nodes can be achieved with low energy overhead. 

\section{Relational Databases for WSN Applications}
Abstractions and unique design paradigms can guide the development of complex systems in ways that are simple to understand. Storing information in a database enables access to specific data via query statements. A database abstraction can be used to simplify the complexities of data collection from a WSN ~\cite{Madden},~\cite{Diallo},~\cite{Chagas}. ~\cite{Madden} uses a WSN design paradigm where the data from each node can be aggregated at a data sink, and the sink acquires the data from each node through SQL-like queries. The nodes act like individual databases, and therefore the complexity of distributed data collection can be managed through simple queries. In this way, data is acquired on demand rather than continuously streaming towards the sink, effectively reducing the amount of data transmitted to only what is desired as opposed to all available samples. The authors in~\cite{Chagas} utilize this approach for data acquisition as well.\\

Database abstractions provide data archival, organization, and access features that support a variety of applications. These features can greatly improve the efficiency and usefulness of WSN systems by providing scalable storage solutions across a variety of platforms. The division of data into multiple tables which can be managed separately and linked via foreign key constraints allows complex yet versatile access by experimenters and analysts. These ideas also pair well with modern software design paradigms such as behavioral parameterization~\cite{urma2014java}, where methods can be written in such a way that they define different behaviors to be decided at runtime. Behavioral parameterization is a powerful design paradigm that will allow WSN data to be easily and accessed and utilized in the ever-changing landscape of IoT applications.\\  

The aim of this thesis is the use of a database abstraction to simplify the complexity of network reconfiguration by storing network state information and other parameters that specify WSN execution, and utilizing this information to generate control messages for dissemination into the network for reconfiguration. Modular software design principles such as behavioral parameterization are used extensively in this thesis to accommodate rapid network reconfiguration as well as changing design requirements of future applications. This approach is described in further detail in Chapter 4.